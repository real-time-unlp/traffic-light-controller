% !TeX spellcheck = es_ES
\documentclass[a4paper,11pt]{article}
\usepackage[spanish]{babel}
\usepackage[left=2cm,right=2cm]{geometry}
\usepackage[utf8]{inputenc}
\usepackage{times}
%opening
\title{Propuesta de proyecto
		\\Sistemas de Tiempo Real 2017
		\\{\Large Eje de giro controlado por PID}
}
\author{García, Agustín Manuel
\\Romero, Ramiro
\\Ternouski, Nahuel}

\date{}

\begin{document}

\maketitle

La idea es controlar el ángulo de giro de una barra conectada a un motor DC. El control es a lazo cerrado, en tiempo real, se monitorea la posición absoluta de la barra con un potenciómetro o preferentemente un encoder incremental. La barra tiene libertad de movimiento de manera que puede caer hacia los costados salvo que se aplique una acción de control. 

\section*{Materiales}
\begin{itemize}
	\item Motor DC de 18V.
	\item Potenciómetro/Encoder incremental de suficiente precisión
	\item La barra a controlar
	\item Varias partes plásticas que acoplen las cosas
	\item Fuentes (a determinar)
	\item Arduino Mega
	\item Fuente de 18V de al menos 2A (a determinar)
	\item Driver  L298 para puente H
	\item Regulador de tensión (a determinar, dependería del micro y fuente)
	\item Componentes pasivos y diodos.
\end{itemize}

\section*{}
Estos materiales son una primera aproximación. De aprobarse este proyecto se armaría una lista más detallada con los modelos y precios de los elementos. 


\end{document}
