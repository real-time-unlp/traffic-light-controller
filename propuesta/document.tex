% !TeX spellcheck = es_ES
\documentclass[a4paper,11pt]{article}
\usepackage[spanish]{babel}
\usepackage[left=2cm,right=2cm]{geometry}
\usepackage[utf8]{inputenc}
\usepackage{times}
%opening
\title{Propuesta de proyecto
		\\Sistemas de Tiempo Real 2017
		\\{\Large Semáforos inteligentes}
}
\author{García, Agustín Manuel
\\Romero, Ramiro
\\Ternouski, Nahuel}

\date{}

\begin{document}

\maketitle

La propuesta se basa en diseñar semáforos inteligentes capaces de detectar los autos que llegan a las intersecciones y, en función de eso, tomar una decisión de cuál es el que tiene prioridad de paso. Con el fin de añadir aún más funcionalidades, los semáforos se equipan con sensores de radiofrecuencia que permiten detectar ambulancias,  policías y bomberos para darles prioridad de paso. Para ello los autos con prioridad deben tener el emisor RF adecuado. Por último, los peatones pueden apretar un botón, ubicado en cada semáforo que les permite generar un pedido de cruce.

\section*{Materiales}
\begin{itemize}
	\item Arduino UNO o Mega.
	\item Red Can-BUS
	\item Sensores ultrasónicos
	\item Pulsadores
	\item Sensores RF
\end{itemize}

\end{document}
